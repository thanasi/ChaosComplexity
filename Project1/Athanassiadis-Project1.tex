%%%%%%%%%%%%%%%%%%%%%%%%%%%%%%%%%%%%%%%%%%%%%%
%Lab report writeup based on template by Derek Hildreth
%%%%%%%%%%%%%%%%%%%%%%%%%%%%%%%%%%%%%%%%%%%%%%

%\documentclass[aps,letterpape,10pt]{revtex4}
\documentclass[aps,letterpaper,10pt]{article}
%\documentclass{article}

\usepackage{graphicx} % For images
\usepackage{float}    % For tables and other floats
\usepackage{verbatim} % For comments and other
\usepackage{amsmath}  % For math
\usepackage{amssymb}  % For more math
\usepackage{fullpage} % Set margins and place page numbers at bottom center
\usepackage{subfig}   % For subfigures
\usepackage[usenames,dvipsnames]{color} % For colors and names
\usepackage{fancyhdr} %headers
\usepackage{listings} %for code
\usepackage{color} %to color code
\usepackage{wrapfig} % for inline images

%Color and code setup
\definecolor{dkgreen}{rgb}{0,0.6,0}
\definecolor{gray}{rgb}{0.5,0.5,0.5}
\definecolor{mauve}{rgb}{0.58,0,0.82}
\definecolor{codebg}{rgb}{.95,.95,.98}

\lstset{ %
	language=Java,
	tabsize=4, 
	numbers=left,
	numberstyle=\footnotesize,
	backgroundcolor=\color{codebg},
	breaklines=true,
	breakatwhitespace=true,
	basicstyle=\small,
	numberstyle=\tiny\color{black},
	showstringspaces=false,
	keywordstyle=\color{blue}, 
	stringstyle=\color{dkgreen},
	commentstyle=\color{gray},
	frame=single,
	title = \texttt{\lstname}
	}

%%%%%%%%%%%%

%HEADER FORMATING%%%%%%%%%%%%%
\pagestyle{fancy}
\headheight 10pt
\setlength{\headsep}{20pt}
\lhead{PHYS 251 - Prof. Tom Witten \\ Project 1}
\rhead{A. Athanassiadis\\Due 10/8/2012}
%%%%%%%%%%%%%%%%%%%%%%%%

%Custom Definitions%%%%%%%%%%%%%%%
\newcommand{\ttt}{\texttt}
%%%%%%%%%%%%%%%%%%%%%%%%

\begin{document}
\section{Problem 1}
\begin{figure}[!h]
\centering
\subfloat[Original Image]{\label{fig:5-1a}\includegraphics[width=.30\textwidth]{img_distance.png}}\hspace{20px}
\subfloat[Distance-Mapped Image]{\label{fig:5-1b}\includegraphics[width=.30\textwidth]{5-1a.png}} \hspace{20px}
\subfloat[Overlay Image]{\label{fig:5-1c}\includegraphics[width=.30\textwidth]{5-1b.png}}
\caption{2-Pass Distance Transform}
\label{fig:5-1}
\end{figure}

In this problem, I applied the two-pass distance transformation as described in class.  Figure \ref{fig:5-1a} shows the original image.  Figure \ref{fig:5-1b} shows the output of the output of the distance transform.  Figure \ref{fig:5-1c} shows an overlay of the two to confirm appropriate output.  The code used for this problem is contained in \ttt{problem1.py} and \ttt{distT.py}

\lstinputlisting{problem1.py}
\newpage
\lstinputlisting{distT.py}

\end{document} 
